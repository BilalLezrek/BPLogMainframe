%%=============================================================================
%% Methodologie
%%=============================================================================

\chapter{\IfLanguageName{dutch}{Methodologie}{Methodology}}%
\label{ch:methodologie}

In dit hoofdstuk stellen we de verwachtingen van MFPOSS vast. Dit doen we zodat het prototype geëvalueerd kan worden zodra het is voltooid. Dit wordt het best gedaan aan de hand van de MoSCoW-methode. Deze methode houdt in dat van tevoren wordt bepaald wat MFPOSS wil van het product, en deze criteria worden verdeeld in de volgende categorieën:

\begin{itemize}
    \item Must have
    \item Should have
    \item Could have
    \item Won't have
\end{itemize}

\section{MoSCoW}

\subsection{Must have}
Voor MFPOSS is het belangrijk dat er eenvoudig te zien is welke berichten vaak voorkomen en welke zelden voorkomen.

\subsection{Should have}
Het is ook de taak van ELK om het leven voor MFPOSS makkelijker te maken zonder dat er te veel extra werk aan besteed moet worden. Het moet dus eenvoudig zijn om elementen aan het dashboard toe te voegen zonder veel programmeerwerk. Het is ook belangrijk dat MFPOSS afwijkingen kan detecteren. Deviatiedetectie kan hierbij goed helpen.

\subsection{Could Have}
Het zou interessant zijn om te onderzoeken of er automatisch meldingen naar MFPOSS kunnen worden gestuurd als ELK iets ongewoons opmerkt.

\subsection{Won't Have}
Een complexe implementatie waaruit te weinig informatie kan worden gehaald. Daarnaast mag het niet te veel extra werk vergen om zaken aan te passen of toe te voegen.

Nadat een prototype is opgesteld, zal worden beoordeeld of het voldoet aan de vooraf vastgestelde MoSCoW-criteria. Als het prototype niet ten minste aan de "Must have" en "Won't have" criteria voldoet, betekent dit dat het voor MFPOSS niet voldoende is om verder mee te werken.
