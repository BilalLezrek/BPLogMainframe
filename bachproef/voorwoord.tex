%%=============================================================================
%% Voorwoord
%%=============================================================================

\chapter*{\IfLanguageName{dutch}{Woord vooraf}{Preface}}%
\label{ch:voorwoord}

%% TODO:
%% Het voorwoord is het enige deel van de bachelorproef waar je vanuit je
%% eigen standpunt (``ik-vorm'') mag schrijven. Je kan hier bv. motiveren
%% waarom jij het onderwerp wil bespreken.
%% Vergeet ook niet te bedanken wie je geholpen/gesteund/... heeft

Deze bachelorproef werd gemaakt in samenwerking met Colruyt Group. Specifieker was diet gedaan met MFPOSS, een systeem beheerders team van de mainframe. De intentie van deze bachelorproef is deels het testen van MFPOSS hun verwachtingen tegenover ELK. Maar ook die van toekomstige mainframe team dat ook een manier zoeken om hun logbestanden beter te kunnen analyseren.

Een grote motivatie voor deze bachelorproef is om een steentje te kunnen bijdragen aan de onderhoudbaarheid van een mainframe. Ik zie de mainframe als verre van iets dat binnenkort gaat uitsterven en een cruciaal deel van onze samenleving. Dus kunnen helpen door log analyse te versimpelen en verduidelijken aan de hand van ELK klinkt zeer interessant. Daarnaast helpt dit de mainframe ook verder moderniseren, iets wat het continue aan het doen is. Daarnaast ben ik ook zelf een voltijdse werknemer van Colruyt Group. Ik werk voor het IDBAMF-team als junior system engineer, specifieker een database administrator op de mainframe. Ik ben dit pas te weten gekomen na dat ik het onderwerp had gekozen, maar IDBAMF is ook aan het kijken naar een ELK integratie. Dus dit onderzoek kunnen doen heeft mij veel ervaring opgeleverd dat ik kan herbruiken binnen mijn eigen team.

Eerst en vooral wil ik mijn promoter en co-promoter bedanken. Dit zijn respectievelijk Jan Willem en Dieter Lefere. Die mij bijzonder goed hebben gesteund doorheen heel het process. Daarnaast zijn er ook een hele lijst aan mensen waar ik mee in contact ben geweest doorheen het onderzoek dat ik wil bedanken voor hun hulp. Dit zijn Steven Goedertier en Arnour Schepens dat mij het contactpunt voor ELK hebben laten zien. Frederik van Brabander dat mij de mogelijkheid heeft gegeven omtijd te maken voor het onderzoek. Jeroen Cornelis dat dieter zijn plaats heeft overgepakt toen hij even niet beschikbaar was. Heel het ELk-team voor hun hulp en uitleg. Mijn vrienden en familie voor hun mentale steun.